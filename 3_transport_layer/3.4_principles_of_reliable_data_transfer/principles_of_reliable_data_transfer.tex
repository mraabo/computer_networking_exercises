\subsection{Principles of Reliable Data Transfer}


\subsubsection{In our \texttt{rdt} protocols, why did we need to introduce sequence numbers? (R9)}

The introduction of sequence numbers are need to provide an identifier to the packets, so that the receiver can identify duplicate packets. Gaps in sequence numbers also makes it possible to identify lost packets.


\subsubsection{In our \texttt{rdt} protocols, why did we need to introduce timers? (R10)}

Timers are needed to establish for the sender when to resend lost or corrupted packets. This implementation is motivated by the possibility of losing ACKs or NAKs.


\subsubsection{Suppose that the roundtrip delay between sender and receiver is constant and known to the sender. Would a timer still be necessary in protocol \texttt{rdt 3.0}, assuming that packets can be lost? Explain. (R11)}

Yes the timer is still needed to monitor how much time has passed. When the RTT is constant it is known to the sender that an ACK or NAK has been lost if 2RTT has passed since sending a packet, and a timer is needed to monitor when 2RTT has passed, so that the sender should resend the packet. The only advantage of having a known an constant RTT is that the sender can know for sure, when an ACK and NAK has been lost.


\subsubsection{Visit the Go-Back-N interactive animation at the Companion Website. (R12)}

\textbf{a. Have the source send five packets, and then pause the animation before any of the five packets reach the destination. Then kill the first packet and resume the animation. Describe what happens.} \\
Since the first packet is lost, the receiver does not acknowledge any of the received packets since they are out of order. Since no ACKs are received by the sender it resend all five packets after the timeout. \\
\\
\textbf{b. Repeat the experiment, but now let the first packet reach the destination and kill the first acknowledgment. Describe again what happens.} \\
When the ACks arrive, the sender knows that the first packet must have been received, because otherwise we would have the case of (a.) where no ACKs are sent. The sender therefore moves the window as if it had received ACK for the first packet as well. Therefore it plays no role if an ACK is lost if ACKs from succeeding packets are received. \\
\\
\textbf{c. Finally, try sending six packets. What happens?} \\
Only five packets are sent since the window size is 5.


\subsubsection{Repeat R12, but now with the Selective Repeat interactive animation. How are Selective Repeat and Go-Back-N different? (R13)}

\textbf{a. Have the source send five packets, and then pause the animation before any of the five packets reach the destination. Then kill the first packet and resume the animation. Describe what happens.} \\
The receiver receives the last four packets, buffers them and sends ACKs for these. The sender gets the ACKs but does not move the sender window since an ACK for the first packet is still missing. After a timeout the sender sends the first packet, which is received by the receiver and then the first packet and the buffered last four packets are send to the application in the correct order. \\
\\
This is different from GBN since the received packets are buffered by the receiver even though they are out of order. However as in GBN the five packets are sent to the application at the same time, since this only happens when the first packet finally arrive. \\
\\
\textbf{b. Repeat the experiment, but now let the first packet reach the destination and kill the first acknowledgment. Describe again what happens.} \\
The five packets are send and received so all packets are delivered to the application. However since the ACK for the first package is not received by the sender, the sender does not move its window. After the timeout the sender retransmits the first packet and when received by the receiver it is rejected as its a duplicate but an ACK for the first packet is retransmitted. When the ACK for the first packet is finally received by the sender, it moves its window to the next five packets. \\
\\
This is different from GBN where it is known, because of the cumulative acknowledgment in GBN, that the first packet must have been received. So where GBN knows that the first packet was received, dispute the missing ACK, and therefore moves its window. On the other hand SR waits for a timeout and retransmits packet one, since SR does not know that the first packet was received as it relies on individual ACKs. SR therefore only moves its window when it finally receives the ACK for packet 1.\\
\\
\textbf{c. Finally, try sending six packets. What happens?} \\
No difference, still only five packets are sent because the window size if 5.