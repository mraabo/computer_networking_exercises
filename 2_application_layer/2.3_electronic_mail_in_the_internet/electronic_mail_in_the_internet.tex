\subsection{Electronic Mail in the Internet}

\subsubsection{What does a stateless protocol mean? Is IMAP stateless? What about SMTP? (R11)}

A stateless protocol is one that does not require the server to maintain information about the user. IMAP is used to communicate with mailservers and is not a stateless protocol, since everything it transfers is maintained in a mail server (emails, folders flags, etc.) SMTP on the other hand is stateless as it functions as a protocol used for pushing email from on mail server to another. SMTP treats each transfer as an independent transaction and does not require any information maintained about previous messages sent.



\subsubsection{Are there any constraints on the format of the HTTP body? What about the email message body sent with SMTP? How can arbitrary data be transmitted over SMTP? (R15)}

Yes, the format of the HTTP body is states in the header of the HTTP message, so the receiver knows what to expect and how to interpret the body. Such formatting requirements could be the content-type (e.g. XML or JSON), encoding and length. For SMTP certain formatting requirements must be satisfied in the body; only plaintext and MIME-encoded content (used for pictures and attachments etc., which i interpret is what is meant by arbitrary data).



\subsubsection{Suppose Alice, with a Web-based e-mail account (such as Hotmail or Gmail), sends a message to Bob, who accesses his mail from his mail server using IMAP. Discuss how the message gets from Alice's host to Bob's host. Be sure to list the series of application-layer protocols that are used to move the message between the two hosts. (R16)}

Alice's host sends the message to her mail server over HTTP. Alice's mailserver then sends the message to Bob's mailserver over SMTP. Bob then transfers the message from his mail server to his host over HTTP. 

\subsubsection{Print out the header of an e-mail message you have recently received. How many \texttt{Received}: header lines are there? Analyze each of the header lines in the message. (R17)}

I received 5 header lines. 
\begin{verbatim}
Received: by 2002:a54:3405:0:b0:229:db56:e5cb with SMTP id l5csp1352389ecq;
    Mon, 21 Aug 2023 03:07:57 -0700 (PDT)
\end{verbatim}
\begin{verbatim}
X-Received: by 2002:a05:6e02:973:b0:34b:aebd:a512 with SMTP id
    q19-20020a056e02097300b0034baebda512mr7254811ilt.14.1692612477236;
\end{verbatim}
\begin{verbatim}
Received: from o1.email.toggl.com (o1.email.toggl.com. [167.89.11.255])
    by mx.google.com with ESMTPS id j2-20020a63e742000000b00563dde13952si6802499pgk.
    720.2023.08.21.03.07.56
    for <magnusraabo@gmail.com>
    (version=TLS1_3 cipher=TLS_AES_128_GCM_SHA256 bits=128/128);
    Mon, 21 Aug 2023 03:07:57 -0700 (PDT)
\end{verbatim}
\begin{verbatim}
Received-SPF: pass (google.com: domain of bounces+99121-6f98-magnusraabo=gmail.com@em4938.
    track.toggl.com designates 167.89.11.255 as permitted sender) client-ip=167.89.11.255;
\end{verbatim}
\begin{verbatim}
Received: by filterdrecv-7bd4cff9b4-bvd8j with SMTP id
    filterdrecv-7bd4cff9b4-bvd8j-1-64E3377C-55
    2023-08-21 10:07:56.541415726 +0000 UTC m=+8850582.239888199
\end{verbatim}
\begin{verbatim}
Received: from smtp-relay-75546b4bf5-7jcs4.localdomain (unknown) by geopod-ismtpd-15 (SG) 
    with ESMTP id o0jxJsdySjqk0c81t6dxIA for <magnusraabo@gmail.com>; Mon, 21 Aug 2023 10:07:56.419 +0000 (UTC)
\end{verbatim}
The header line above indicates the id of the SMTP server that sends and stores the email in my SMTP server.