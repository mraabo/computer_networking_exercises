\subsection{Networks Under Attack}

\subsubsection{What is self-replicating malware? (R26)}
Malware that once present on a computer can download more malware on this computer or spread itself to other computers from the infected one.


\subsubsection{Describe how a botnet can be created and how it can be used for a DDoS attack. (R27)}
A botnet is created by spreading malware on several computers, so that the hacker can control all of these computers. These can be used for a DDoS attack in the following ways:
\begin{itemize}
    \item \textbf{Bandwidth flooding}: Sending so many packets to the targeted host, that the access link of the host becomes clogged, preventing legitimate packets from reaching the host.
    \item \textbf{Connection flooding}: Sending repeated bogus requests for connection to the targeted host, causing the targeted host to stop accepting legitimate requests for connection.
\end{itemize}



\subsubsection{Suppose Alice and Bob are sending packets to each other over a computer network. Suppose Trudy positions herself in the network so that she can capture all the packets sent by Alice and send whatever she wants to Bob; she can also capture all the packets sent by Bob and send whatever she wants to Alice. List some of the malicious things Trudy can do from this position. (R28)}
Trudy can do several malicious things, with the following being only a few examples:
\begin{itemize}
    \item She can quietly intercept all packets and "sniff" on the packet being sent, if she can decrypt them. Perhaps send them to a third party.
    \item She can pretend to be Bob to Alice and pretend to be Alice to Bob, and send fake messages to both of them.
    \item She can stop the packets from reaching either Bob or Alice or both and thus prevent their communication.
    \item She can perform a combination of the above.
\end{itemize}