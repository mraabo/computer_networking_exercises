\subsection{The Network Edge}

\subsubsection{List four access technologies. Classify each one as home access, enterprise access, or wide-area wireless access. (R4)}

\begin{enumerate}
    \item iPhone: wide-area wireless access
    \item Stationary PC: home access
    \item Personal Laptop: home access
    \item Work LapTop: enterprise access
\end{enumerate}



\subsubsection{Is HFC transmission rate dedicated or shared among users? Are collisions possible in HFC? Why or why not? (R5)}
Hybrid fiber coax (HFC) is a combination of optical fiber and coaxial cable. The optical fiber is used for most of the length and connects neighborhood junctions, from which coaxial follows and connects to individual houses. On the downstream channel, all packets come from the same source, the head end of the fiber cable and therefore the users share the HFC bandwidth. This also means that there are no collisions.



\subsubsection{What access network technologies would be most suitable for providing internet access in rural areas? (R6)}
Rural areas are likely to be far from internet providers, which will typically lie ind the city. Therefore wireless technologies are the most suitable. As such terrestrial radio channels or satellite radio channels might provide the best solution.



\subsubsection{Dial-up modems and DSL both use the telephone line (a twisted-pair copper cable) as their transmission medium. Why then is DSL much faster than dial-up access? (R7)}

DSL is much faster than dial-up because it offers wider bandwidth (uses a wider frequency range), digital transmission (as opposed to analog, which is more susceptible to noise interference), multiple frequencies and channels, and asymmetric speed configurations. These factors allow for higher data rates compared to dial-up modems.


\subsubsection{What are some of the physical media that Ethernet can run over? (R8)}
Twisted-pair copper wire, coaxial cable, and fiber-optic cable.


\subsubsection{HFC, DSL, and FTTH are all used for residential access. For each of these access technologies, provide a range of transmission rates and comment on whether the transmission rate is shared or dedicated. (R9)}
\begin{itemize}
    \item HFC: \begin{itemize}
        \item Downstream: 40 Mbps - 1.2 Gbps
        \item Upstream: 30 Mbps - 100 Mbps
        \item Transmission rate: Shared
        \end{itemize}
    \item DSL: \begin{itemize}
        \item Downstream: 24 Mbps - 52 Mbps
        \item Upstream: 3.5 Mbps - 16 Mbps
        \item Transmission rate: Dedicated
        \end{itemize}
    \item FTTH: \begin{itemize}
        \item Downstream: Up to 100+ Gbps
        \item Upstream: Up to 100+ Mbps
        \item Transmission rate: Dedicated
        \end{itemize}
\end{itemize}


\subsubsection{Describe the different wireless technologies you use during the day and their charecteristics. If you have a choice between multiple technologies, why do you prefer one over another? (R10)}
My iPhone which i prefer for tasks that require short attention and can be performed on a small screen. A stationary PC for longer tasks that can benefit from larger and dual screens. Laptop for on-the-go tasks which require a keyboard and a larger screen than the iPhone. 